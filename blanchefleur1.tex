\documentclass[a4paper,draft]{article}

\usepackage{amsmath,amsthm,latexsym,eufrak,amssymb}
\def\rmd{\mathrm{d}}
\def\rme{\mathrm{e}}
\def\PE{\mathbb{E}}

\newtheorem{theorem}{Theorem}[section]
\newtheorem{proposition}[theorem]{Proposition}
\newtheorem{lemma}[theorem]{Lemma}
\newtheorem{corollary}[theorem]{Corollary}
\newtheorem{remark}[theorem]{Remark}
\newtheorem{definition}[theorem]{Definition}
\newtheorem{example}[theorem]{Example}
\newtheorem{assumption}[theorem]{Assumption}

\begin{document}

\title{On recursive estimation schemes\\ with stationary data streams
\thanks{All the authors
were supported
by The Alan Turing Institute, London under the EPSRC grant EP/N510129/1.
M. R. also enjoyed the support of the NKFIH (National Research, Development and Innovation Office, Hungary)
grant KH 126505 and the ``Lend\"ulet'' grant LP 2015-6 of the
Hungarian Academy of Sciences.}}

\author{M. Barkhagen \and N. H. Chau \and \'E. Moulines \and
M. R\'asonyi \and S. Sabanis \and Y. Zhang}

\date{\today}

\maketitle

\begin{abstract}
red
\end{abstract}

\section{Introduction}

We are concerned with sampling from the distribution $\pi$ defined by
$$
\pi(A):=\int_A \rme^{-U(x)}\, \rmd x/\int_{\mathbb{R}^d} \rme^{-U(x)}\, \rmd x,\
A\in\mathcal{B}(\mathbb{R}^d),
$$
where $\mathcal{B}(\mathbb{R}^d)$ denotes the Borel sigma-field
and $U:\mathbb{R}^d\to\mathbb{R}_+$ is continuously
differentiable. We assume, for simplicity,
that $U(0)=0$. The sampling algorithms will be based on
the derivative $h:=\nabla U$ and on its noisy observations.
We only treat the case of unbiased observations but a bias
could also easily be incorporated.

\section{Main results}

We are working on a probability space $(\Omega,\mathcal{F},P)$.
Expectation of
a random variable $X$ will be denoted by $\PE[X]$.
For any $m\geq 1$, for any $\mathbb{R}^m$-valued random variable $X$ and for any $1\leq p<\infty$, let us set
$\Vert X\Vert_p:=\sqrt[p]{\PE[|X|^p]}$. We denote by $L^p$ the set of $X$ with $\Vert X\Vert_p<\infty$.
The indicator function of a set $A$ will be denoted by $1_A$.


Let $\theta_0$ be an $\mathbb{R}^d$-valued random variable, representing
the initial value of the procedures we consider.
Let $H:\mathbb{R}^d\times\mathbb{R}^m\to\mathbb{R}^d$ be a measurable
function, let $X_t$, $t\in\mathbb{Z}$ be an $\mathbb{R}^m$-valued (strict sense) stationary process.
Let $\xi_t$, $t\in\mathbb{N}$ be an independent
sequence of standard Gaussian random variables. We assume that $\theta_0$, $(X_t)_{t\in\mathbb{Z}}$,
$(\xi_t)_{t\in\mathbb{N}}$ are independent.

For each $\lambda>0$, define the $\mathbb{R}^d$-valued
random process $\theta^{\lambda}_t$, $t\in\mathbb{N}$ by recursion:
\begin{equation}\label{nab}
\theta^{\lambda}_0:=\theta_0,\quad \theta^{\lambda}_{t+1}:=\theta^{\lambda}_t-\lambda H(\theta^{\lambda}_t,X_{t+1})+\sqrt{2\lambda}\xi_{t+1}.
\end{equation}

We assume that $h(\theta)=E[H(\theta,X_0)]$, $\theta\in\mathbb{R}^d$
(in particular, the expectations are finite), and go on
defining the ``averaged'' version of \eqref{nab},
\begin{equation}\label{aver}
\overline{\theta}^{\lambda}_0:=\theta_0,\quad
\overline{\theta}^{\lambda}_{t+1}:=\overline{\theta}^{\lambda}_t-\lambda
h(\overline{\theta}^{\lambda}_t)+\sqrt{2\lambda}\xi_{t+1}.
\end{equation}

\begin{assumption}\label{lip} There is $L>0$ such that
$$
|H(\theta_1,x_1)-H(\theta_2,x_2)|\leq L[|\theta_1-\theta_2|+|x_1-x_2|].
$$
Furthermore, $X_0\in L^2$.
\end{assumption}

Under Assumption \ref{lip}, $h$ is also Lipschitz-continuous. Next, monotonicity conditions are required for $H$. Scalar product in $\mathbb{R}^d$
is denoted by $\langle \cdot,\cdot\rangle$.

\begin{assumption}\label{diss} There is a function $a:\mathbb{R}^m\to\mathbb{R}_+$ such that, for all $\theta_1,\theta_2\in\mathbb{R}^d$ and $x\in\mathbb{R}^m$,
\begin{equation}\label{montre}
\langle \theta_1-\theta_2,H(\theta_1,x)-H(\theta_2,x)\rangle\geq
a(x)[|\theta_1-\theta_2|^2 + |H(\theta_1,x)-H(\theta_2,x)|^2]
\end{equation}
and $Ea(X_0)>0$.
\end{assumption}

This assumption clearly holds if, for all $x$, $\theta\to H(\theta,x)$
is the derivative of a strongly convex function.
IN THIS FIRST DRAFT WE ASSUME $a(\cdot)$ TO BE CONSTANT. THIS CAN BE
RELAXED MODULO SOME EXTRA WORK.

For technical purposes it is convenient to assume a specific structure
for the probability space and the filtrations we consider.

\begin{assumption}\label{opa}
Let $\mathcal{X}$ be a Polish space.
We assume that $\Omega=(\mathcal{X}\times \mathbb{R}^d)^{\mathbb{Z}}$, $\mathcal{F}$ consists of the Borel sets of $\Omega$ and
$P=\otimes_{i\in\mathbb{Z}} \psi$ where $\psi=\nu\otimes\varpi$ with $\nu$ a fixed probability measure on $\mathcal{X}$ and $\varpi$ standard Gaussian
on $\mathbb{R}^d$.
The coordinate mappings from $\Omega$ to $\mathcal{X}$ will be denoted by $\Lambda_i=(\varepsilon_i,\xi_i)$, $i\in\mathbb{Z}$, where
$\xi_i$, $i\geq 1$ are the random variables figuring in \eqref{nab} and
\eqref{aver} and $X_t=g(\varepsilon_t,\varepsilon_{t-1},\ldots)$
with a fixed measurable function $g:\mathcal{X}^{-\mathbb{N}}\to\mathbb{R}^m$.
We furthermore set
$\mathcal{G}_n:=\sigma(\varepsilon_i,\, i\leq n)$, as well as $\mathcal{G}^+_n:=\sigma(\varepsilon_i,\, i>n)$,
for each $n\in\mathbb{N}$. Define also $\mathcal{F}_n:=\mathcal{G}_n\vee
\sigma(\xi_i,\ i\in\mathbb{N})$ and $\mathcal{F}_n^+:=\mathcal{G}_n^+$.
\end{assumption}



Our aim is to estimate
$\Vert\theta^{\lambda}_t-\overline{\theta}^{\lambda}_t\Vert_2$,
uniformly in $t$.

\begin{example} {\rm Let $H(\theta,x):=\theta+x$ and let $X_n$, $n\in\mathbb{N}$
be an independent sequence of standard Gaussian random variables,
independent of $\xi_n$, $n\in\mathbb{N}$. Take $\theta_0:=0$.
It is straightforward to check that
$$
\overline{\theta}^{\lambda}_t-\theta^{\lambda}_t=\sum_{j=0}^{t-1}
(1-\lambda)^j \lambda X_{t-j}
$$
which clearly has variance
$$
\sum_{j=0}^{t-1}(1-\lambda)^{2j}\lambda^2=\frac{\lambda(1-(1-\lambda)^{2t})}
{2-\lambda}.
$$
It follows that
$$
\sup_{t\in\mathbb{N}}||\overline{\theta}^{\lambda}_t-\theta^{\lambda}_t||_2=\sqrt{\frac{\lambda}{2-\lambda}}.
$$
This shows that the best estimate we may hope to get is of the
order $\sqrt{\lambda}$.
Our Theorem \ref{main} below achieves this bound modulo a logarithmic
factor (which does not seem to matter in practice).}
\end{example}

\begin{theorem}\label{main} Let $X$ be conditionally $L$-mixing
of order $(2,4)$ with respect to $(\mathcal{G}_t,\mathcal{G}_t^+)$.
Let Assumptions \ref{lip}, \ref{diss} and
\ref{opa} hold. Then there exists $C^{\circ}>0$ such that
\begin{equation}
||\theta^{\lambda}_t-\overline{\theta}^{\lambda}_t||_2
\leq C^{\circ}\sqrt{\lambda}|\ln(\lambda)|^{3/2},\ t\in\mathbb{N}.
\end{equation}
\end{theorem}

The next corollary relates our findings in Theorem \ref{main}
to the problem of sampling from $\pi$. Let $W_2$
denote the Wasserstein metric of order $2$, see e.g. \cite{villani}
for more information about this distance.

\begin{corollary}\label{dros} For each $\kappa>0$, there exist constants $c_1,c_2>0$ such that, for each
$\epsilon>0$ one has
$$
W_2(\mathrm{Law}(\theta^{\lambda}_t),\pi)\leq \epsilon
$$
whenever
\begin{equation}\label{sharp}
\lambda\leq  c_1\epsilon^{2+\kappa}\mbox{ and }t\geq
\frac{c_2}{\epsilon^{2+\kappa}}\ln(1/\epsilon).
\end{equation}
\end{corollary}

\begin{remark}{\rm Corollary \ref{dros} significantly improves on
some of the results in \cite{raginsky} in certain cases, compare
also to \cite{xu}.
In \cite{raginsky} the monotonicity assumption \eqref{montre} is not
imposed, only a dissipativity condition is required and a more
general recursive scheme is investigated. However, the input
sequence $X_t$, $t\in\mathbb{N}$ is assumed i.i.d.
In that setting, Theorem of
\cite{raginsky} applies to \eqref{nab} (with the choice $\delta=0$,
$\beta=1$, $d$ fixed, see also the last paragraph of Subsection 1.1
of \cite{raginsky}), and we get that
$$
W_2(\mathrm{Law}(\theta^{\lambda}_t),\pi)\leq \epsilon
$$
holds whenever
$\lambda\leq c_3(\epsilon/\ln(1/\epsilon))^4$ and
$t\geq \frac{c_4}{\epsilon^4}\ln^5(1/\epsilon)$ with some $c_3,c_4>0$.
Our results provide the sharper estimates \eqref{sharp}. The main
purpose of the present note is to provide results for the case where
$X_t$, $t\in\mathbb{N}$ has dependencies, but \eqref{sharp} is
new even in the particular case where $X_t$, $t\in\mathbb{N}$ are i.i.d.}
\end{remark}

\section{Conditional $L$-mixing}\label{lm}

$L$-mixing processes and random fields
were introduced in \cite{laci1}. They proved to be useful in
tackling difficult problems of system identification. In
\cite{4} the related concept of \emph{conditional} $L$-mixing
was introduced in order to treat fixed gain recursive estimators
with discontinous updating functions. Although the function $H$
is assumed continuous in the present article, it seems that
the right setting for the analysis of \eqref{nab} is provided
by conditionally $L$-mixing random fields, which we will define below.

We assume that the probability space is equipped
with a discrete-time filtration $\mathcal{H}_n$, $n\in\mathbb{N}$ as well as with a decreasing sequence of sigma-fields $\mathcal{H}_n^+$, $n\in\mathbb{N}$ such that $\mathcal{H}_n$ is
independent of $\mathcal{H}_n^+$, for all $n$.

Fix an integer $d\geq 1$ and let $D\subset \mathbb{R}^d$ be a set of parameters. A measurable function
$X:\mathbb{N}\times D\times\Omega\to\mathbb{R}^m$ is called a random field. We will drop dependence on $\omega\in\Omega$ and
use the notation $X_t(\theta)$, $t\in\mathbb{N}$, $\theta\in D$. A random
process $X_t$, $t\in\mathbb{N}$ corresponds to a random field where $D$
is a singleton. A random field is $L^r$-\emph{bounded} for some $r\geq 1$
if
$$
\sup_{t\in\mathbb{N}}\sup_{\theta\in D} ||X_t(\theta)||_r<\infty.
$$

Now we define conditional $L$-mixing.
Recall that, for any family $Z_i$, $i\in I$ of real-valued random variables, $\mathrm{ess.}\sup_{i\in I} Z_i$
denotes a random variable that is an almost sure upper bound for each $Z_i$ and it is a.s.
smaller than or equal to any other such bound, see e.g. Proposition VI.1.1. of \cite{neveu}.


Let $X_t(\theta)$, $t\in\mathbb{N}$, $\theta\in D$ be a random field
bounded in $L^r$.
Define, for each $n\in\mathbb{N}$,
\begin{eqnarray*}
	M^{n}_r(X) &:=& \mathrm{ess}\sup_{\theta\in D}\sup_{t \in\mathbb{N}}
	E^{1/r}[|X_{n+t}(\theta)|^r\big\vert\mathcal{H}_n],\\
	\gamma^{n}_r(\tau,X)&:=& \mathrm{ess}\sup_{\theta\in D}\sup_{t\geq\tau}
	E^{1/r}[|X_{n+t}(\theta)-E[X_{n+t}(\theta)\vert \mathcal{H}_{n+t-\tau}^+\vee \mathcal{H}_n]|^r\big\vert
	\mathcal{H}_n],\ \tau\geq 1,\\
	\Gamma^{n}_r(X) &:=&\sum_{\tau= 1}^{\infty}\gamma^{n}_r(\tau,X).
\end{eqnarray*}
When necessary, we will also use the notations $M^{n}_r(X,D)$,
$\gamma^{n}_r(\tau,X,D)$, $\Gamma^{n}_r(X,D)$ to signal dependence of
these quantities on the domain $D$ which may vary.

For some $r,p\geq 1$, we call $X_t(\theta)$, $t\in\mathbb{N}$, $\theta\in D$
\emph{uniformly {conditionally} $L$-mixing of order $(r,p)$} (UCLM-$(r,p)$)
with respect to $(\mathcal{H}_t,\mathcal{H}_t^+)$ if
it is $L^r$-bounded; $X_t(\theta)$, $t\in\mathbb{N}$ is adapted to
$\mathcal{F}_t$, $t\in\mathbb{N}$
for all $\theta\in D$
and the sequences  $M^n_r(X)$, $\Gamma^n_r(X)$, $n\in\mathbb{N}$
are bounded in $L^p$. In the case of stochastic processes (when $D$ is a singleton)
the terminology ``conditionally $L$-mixing process of order $(r,p)$'' will be used.


The following maximal inequality is pivotal for our arguments.

\begin{theorem}\label{estim} Let Assumption \ref{opa} be in force.
Fix $r>2$, $n\in \mathbb{N}$.
Let $W_t$, $t\in\mathbb{N}$ be a conditionally L-mixing process of order
$(r,1)$ w.r.t. $(\mathcal{H}_t,\mathcal{H}_t^+)$, satisfying
$E[W_t\vert\mathcal{H}_n]=0$ a.s. for all {$t\geq n$}.
Let $m >n$ and let $b_t$, $n< t\leq m$ be deterministic numbers. Then we have
\begin{equation}\label{mandrill}
E^{1/r}\left[ \sup_{n < t \le m} \left| \sum_{s = n+1}^{t} b_s W_s \right|^r \big\vert\mathcal{H}_n \right]
 \le C_r \left( \sum_{s=n+1}^{m} b_s^2 \right)^{1/2} \sqrt{{M}_r^{n}(W) \Gamma_r^{n}(W)},
\end{equation}
almost surely, where $C_r$ is a deterministic constant depending only on $r$ but independent of $n,m$.
\end{theorem}

For each $R\geq 0$
we denote $B(R):=\{x\in\mathbb{R}^d:\, |x|\leq R\}$, the closed
ball of radius $R$ around the origin.

\begin{lemma}\label{below} Let $X_t$, $t\in\mathbb{N}$ be UCLM-$(r,p)$. Let Assumption \ref{lip} hold true. Then,
for each $j\in\mathbb{N}$, the random field $H(\theta,X_t)$,
$t\in\mathbb{N}$, $\theta\in B(j)$ is UCLM-$(r,p)$.
\end{lemma}
\begin{proof} Notice that
\begin{eqnarray*}
|H(\theta,x)|\leq |H(\theta,x)-H(\theta,0)|+
|H(\theta,0)-H(0,0)|+ |H(0,0)| &\leq&\\
L|x| + L|\theta|+|H(0,0)|.
\end{eqnarray*}
Let $\theta\in B(j)$. Then, for $k\geq n$,
$$
E[|H(\theta,X_k)|^r\vert\mathcal{H}_n]\leq
C(r)[E[|X_k|^r\vert\mathcal{H}_n]+j^r+1]
$$
for some $C(r)>0$ hence
$$
M^n_r(H(\theta,X),B(j))\leq C^{1/r}(r)[M_r^n(X)+j+1].
$$
We also have
\begin{eqnarray*}
E^{1/r}[|H(\theta,X_k)-E[H(\theta,X_k)\vert\mathcal{H}_n\vee\mathcal{H}_{n-\tau}^+]|^r\vert\mathcal{H}_n] &\leq&\\
2E^{1/r}[|H(\theta,X_k)-H(\theta,E[X_k\vert\mathcal{H}_n\vee\mathcal{H}_{n-\tau}^+])|^r\vert\mathcal{H}_n] &\leq&\\
2L E^{1/r}[|X_k-E[X_k\vert\mathcal{H}_n\vee\mathcal{H}_{n-\tau}^+]|^r
\vert\mathcal{H}_n], & &
\end{eqnarray*}
using Lemma \ref{mall}, which implies
$$
\Gamma^n_r(H(\theta,X),B(j))\leq 2L\Gamma^n_r(X).
$$
\end{proof}


\begin{lemma}\label{mall} Let $\mathcal{G},\mathcal{H}\subset\mathcal{F}$
be sigma-algebras. Let $X,Y$ be random variables in $L^p$ such that $Y$ is measurable with
	respect to $\mathcal{H}\vee\mathcal{G}$.
	Then for any $p\ge 1$,
	$$
	E^{1/p}\left[\vert X-E[X\vert\mathcal{H}\vee\mathcal{G}]\vert^p\big\vert \mathcal{G}\right]
	\leq 2E^{1/p}\left[\vert X-Y\vert^p\big\vert \mathcal{G}\right].
	$$
	If $Y$ is $\mathcal{H}$-measurable then
	\begin{equation}\label{dirk}
	\Vert X-E[X\vert\mathcal{H}]\Vert_p\leq 2 \Vert X-Y\Vert_p.
	\end{equation}
\end{lemma}
\begin{proof} See Lemma 6.2 of \cite{4}.
\end{proof}





\section{Ergodic properties of recursive schemes}

One of the key observations is the following contraction property
of the Markov chain $\overline{\theta}_t$, $t\in\mathbb{N}$.

\begin{lemma}\label{ghibli}
Let $\theta_0,\theta_0'\in L^2$ be random variables and $\xi$ standard
Gaussian, independent of $\sigma(\theta_0,\theta_0')$.
Then there is $\rho>0$
such that, defining
$$
\theta_1:=\theta_0-\lambda h(\theta_0)+\sqrt{2\lambda}\xi,\
\theta_1':=\theta_0'-\lambda h(\theta_0')+\sqrt{2\lambda}\xi,
$$
we have
$$
E[(\theta_1-\theta_1')^2]\leq e^{-\rho\lambda}E[(\theta_0-\theta_0')^2].
$$
\end{lemma}
\begin{proof}
See Proposition 3 of \cite{durmus-moulines}.
\end{proof}

Let $V(x):=\exp(U(x)/2)$.
\begin{lemma}\label{tilda} There exists $\tilde{c}>0$ such that $U(x)/2\geq \tilde{c}x^2$
holds for all $x\in\mathbb{R}^d$.
\end{lemma}

Let us fix $\tilde{c}$ as in Lemma \ref{tilda} and define $\tilde{V}(x):=
\exp(\tilde{c}x^2)$, $x\in\mathbb{R}^d$.


\begin{lemma}\label{toranaga} Let $\chi_n$, $n\geq 1$ be such that
$$
\sup_{n\geq 1} E\tilde{V}(\chi_n)<\infty.$$
Denote $\zeta_n:=\sup_{1\leq k\leq n}|\chi_k|$.
Then, for all $p>0$, and for all $n\geq 1$,
$$
E|\zeta_n|^p\leq \tilde{C}(p)\ln^{p/2}(n+1).
$$
holds for some $\tilde{C}(p)>0$.
\end{lemma}
\begin{proof} Define the \emph{convex} function
$$
f(x):=e^{\tilde{c}|x|^{2/p}},\ |x|\geq \left(\frac{p}{2\tilde{c}}\right)^{p/2},\ f(x):=e^{p/2},\ |x|<\left(\frac{p}{2\tilde{c}}\right)^{p/2}.
$$
Denote $M:=\sup_{n\in\mathbb{N}}E\tilde{V}(\xi_n)$.
Jensen's inequality and trivial considerations show that
\begin{eqnarray*}
f(E|\zeta_n|^p)\leq Ef(|\zeta_n|^p)\leq Ee^{\tilde{c}|\zeta_n|^2}+
e^{p/2} &\leq&\\
e^{p/2}+\sum_{j=1}^n Ee^{\tilde{c}|\chi_j|^2} \leq nM +e^{p/2}.
\end{eqnarray*}
This implies also
$$
e^{\tilde{c}E^{2/p}|\zeta_n|^p}\leq nM +e^{p/2},
$$
which leads to
$$
E|\zeta_n|^p\leq \tilde{C}(p)(\ln(n+1))^{p/2},
$$
for some $\tilde{C}(p)>0$, as stated.
\end{proof}

\begin{lemma} Under Assumption \ref{diss},
$\sup_{n} EV(\theta_n)<\infty$.
\end{lemma}
\begin{proof}
Assumption \ref{diss} implies that $V$ is a Lyapunov function for
this stochastic system (see Proposition 8 of \cite{unadjusted}) and the statement easily follows from this.
\end{proof}

\begin{lemma}\label{lavel} Let Assumption \ref{diss} hold.
We also have $\sup_n EV(\overline{z}_n)<\infty$ and $\sup_n EV(\overline{\theta}_n)<\infty$. {A fortiori},
$\sup_n E\tilde{V}(\overline{z}_n)<\infty$
\end{lemma}

\begin{lemma}\label{easy}
There is $C^{\flat}>0$ such that
$$
\sup_{n}\Vert |H(\overline{\theta}_n,X_{n+1})|+|h(\overline{z}_n|\Vert_2\leq C^{\flat}.
$$
\end{lemma}
\begin{proof}
This is quite trivial from Assumption \ref{lip} and Lemma \ref{lavel}, details later.
\end{proof}

Clearly, since
$X$ is conditionally $L$-mixing of order $(2,4)$ with respect to
$(\mathcal{G}_t,\mathcal{G}^+_t)$, it remains
conditionally $L$-mixing of order $(2,4)$ with respect to
$(\mathcal{F}_t,\mathcal{F}^+_t)$, too.

For each $\theta\in\mathbb{R}^d$, $0\leq s\leq t$, we recursively define
$$
z(s,s,\theta):=\theta,\quad z(t+1,s,\theta):=z(t,s,\theta)
-\lambda h(z(t,s,\theta))+\sqrt{2\lambda}\xi_{t+1}.
$$
We then set, for each $n\in\mathbb{N}$ and for each
$nT\leq t<(n+1)T$, $\overline{z}_t:=z(t,nT,\theta_{nT})$.
Note that $\overline{z}_t$ is then defined for all $t\in\mathbb{N}$
and that $\overline{\theta}_t=z(t,0,\theta_0)$.

\begin{lemma}\label{kaaka}
There is a random variable $\Xi$ such that, for all $\theta\in\mathbb{R}^d$
and for all $n\in\mathbb{N}$,
$$
\sum_{k=nT+1}^{\infty}|h_{k,nT}(\theta)-h(\theta)|\leq \Xi
$$
and $E[\Xi^2]<\infty$.
\end{lemma}
\begin{proof}
Notice that, since $E[X_k\vert\mathcal{F}_{nT}^+]$ is independent
of $\mathcal{F}_{nT}$,
$$
E[H(\theta,E[X_k\vert\mathcal{F}_{nT}^+])\vert\mathcal{F}_{nT}]=
E[H(\theta, E[X_k\vert\mathcal{F}_{nT}^+])].
$$
This implies that
\begin{eqnarray*}
|h_{k,nT}(\theta)-h(\theta)| &\leq&\\
\left|E[H(\theta,X_k)|\mathcal{F}_{nT}]-
E[H(\theta,E[X_k\vert\mathcal{F}_{nT}^+])\vert\mathcal{F}_{nT}]\right|
&+&\\
\left|E[H(\theta, E[X_k\vert\mathcal{F}_{nT}^+])]-E[H(\theta,X_k)]\right| &\leq&\\
LE[|X_k-E[X_k\vert\mathcal{F}_{nT}^+]|\vert\mathcal{F}_{nT}]
+LE[|X_k-E[X_k\vert\mathcal{F}_{nT}^+]|] &\leq&\\
L[\gamma_1^{nT}(X,k-nT) + E\gamma_1^{nT}(X,k-nT)]. & &
\end{eqnarray*}
Hence
$$
\sum_{k=nT+1}^{\infty}|h_{k,nT}(\theta)-h(\theta)|\leq L[\Gamma_1^{nT}(X)
+E\Gamma_1^{nT}(X)].
$$
Since $X$ is conditionally $L$-mixing of order $(2,4)$, it is also
conditionally $L$-mixing of order $(1,2)$ so
$E[(\Gamma_1^{nT}(X))^2]<\infty$, This implies the statement.
\end{proof}


\begin{proof}[Proof of Theorem \ref{main}.] Let $T:=\lfloor 1/\lambda\rfloor$.
Fix $n\in\mathbb{N}$ and let $nT\leq t<(n+1)T$ be arbitrary.
Let us define the (random) functions
$$
h_{t,nT}(\theta):=
E[H(\theta,X_t)\vert\mathcal{F}_{nT}],\ \theta\in\mathbb{R}^d.
$$
It is tedious but standard to show that there exists a jointly measurable version of
$$
(\omega,\theta)\to h_{k,nT}(\theta,\omega),\
(\omega,\theta)\in\Omega\times \mathbb{R}^d.
$$
Estimate,
\begin{eqnarray*}
| \theta_{t}-\overline{z}_{t}| \leq \lambda \left\vert\sum_{k=nT+1}^t
\left(H(\theta_k,X_k)-h(\overline{z}_k)\right)\right\vert &\leq&\\
\lambda \sum_{k=nT+1}^t \vert
H(\theta_k,X_k)-H(\overline{z}_k,X_k)\vert
&+&\\
\lambda \left\vert\sum_{k=nT+1}^{t}
\left(H(\overline{z}_k,X_k)-h_{k,nT}(\overline{z}_k)\right) \right\vert &+&\\
\lambda \sum_{k=nT+1}^t
\left\vert h_{k,nT}(\overline{z}_k)- h(\overline{z}_k)\right\vert &\leq&\\
\lambda L \sum_{k=nT+1}^t |\theta_k-\overline{z}_k| &+&\\
\lambda
\max_{nT+1\leq m< (n+1)T} \left|\sum_{k=nT+1}^m \left(H(\overline{z}_k,X_k)-
h_{k,nT}(\overline{z}_k)\right)\right| &+&\\
\lambda \sum_{k=nT+1}^{\infty} \vert h_{k,nT}(\overline{z}_k)- h(\overline{z}_k)\vert, & &
\end{eqnarray*}
by Assumption \ref{lip}.
Gronwall's lemma and taking squares lead to
\begin{eqnarray*}
|\theta_{t}-\overline{z}_{t}|^2 \leq & & \\
2\lambda^2 e^{2LT\lambda}
\left[\max_{nT+1\leq m< (n+1)T} \left|\sum_{k=nT+1}^m \left(H(\overline{z}_k,X_k)-
h_{k,nT}(\overline{z}_k)\right)\right|^2 \right.
&+& \\
\left. \left(\sum_{k=nT+1}^{\infty} \left\vert h_{k,nT}(\overline{z}_k)- h(\overline{z}_k)\right\vert\right)^2\right], & &
\end{eqnarray*}
noting also $(x+y)^2\leq 2(x^2+y^2)$, $x,y\in\mathbb{R}$.
Let $N$ be the random variable
$N:=\sup_{nT+1\leq i< (n+1)T}|\overline{z}_i|$.
Now, recalling the definition of $T$ and taking $\mathcal{F}_{nT}$-conditional expectations, we can write
\begin{eqnarray*}
E\left[|\theta_{t}-\overline{z}_{t}|^2\vert\mathcal{F}_{nT}\right]
\leq & & \\
2\lambda^2 e^{2L}\left[\sum_{j=1}^{\infty}1_{\{j-1\leq N <j\}}
E\left[\max_{nT+1\leq m<(n+1)T} \left|\sum_{k=nT+1}^m \left(H(\overline{z}_k,X_k)-
h_{k,nT}(\overline{z}_k)\right)\right|^2 \vert\mathcal{F}_{nT}\right]\right.
&+& \\
\left. E\left[\left(\sum_{k=nT+1}^{\infty} \vert h_{k,nT}(\overline{z}_k)-
h(\overline{z}_k)\vert\right)^2\vert\mathcal{F}_{nT}\right]
\right]. & &
\end{eqnarray*}


Using the $\mathcal{F}_{nT}$-measurability of $\overline{z}_k$,
$nT\leq k<(n+1)T$, Lemma \ref{toranaga} and taking expectations, we can continue our estimations as
\begin{eqnarray*}
E| \theta_{t}-\overline{z}_{t}|^2 &\leq&\\
2\lambda^2 e^{2L}\sum_{j=1}^{\infty}E[1_{\{j-1\leq N <j\}}
T\Gamma_2^{nT}(H(\theta,X),B(j))M_2^{nT}(H(\theta,X),B(j))] &+&\\
2\lambda^2 e^{2L}
E\left[\Xi^2\right],
\end{eqnarray*}
see Lemma \ref{kaaka}.
By the Cauchy inequality and the trivial $\{j-1\leq N\}=\{j\leq N+1\}$,
an application of Theorem \ref{estim} gives
\begin{eqnarray*}
\sum_{j=1}^{\infty}E[1_{\{j-1\leq N <j\}}
\Gamma_2^{nT}(H(\theta,X),B(j))M_2^{nT}(H(\theta,X),B(j))] &\leq&\\
\sum_{j=1}^{\infty} P^{1/2}(N+1\geq j)
E^{1/2}[(\Gamma_2^{nT}(H(\theta,X),B(j)))^2(M_2^{nT}(H(\theta,X),B(j)))^2] &\leq&\\
\sum_{j=1}^{\infty}\sqrt{\frac{E(N+1)^6}{j^6}}2LE^{1/2}[(\Gamma^{nT}_2(X))^2 C(2)[M_2^{nT}(X)+j+1]^2] &\leq&\\
\sum_{j=1}^{\infty}\sqrt{\frac{E(N+1)^6}{j^6}}2L\sqrt{C(2)}
E^{1/4}[(\Gamma^{nT}_2(X))^4]
E^{1/4}[M_2^{nT}(X)+j+1]^4] &\leq&\\
\check{C}'\sum_{j=1}^{\infty} \frac{\ln^3(T)}{j^2}
\leq \check{C}\ln^3(T), & &
\end{eqnarray*}
for suitable $\check{C}, \check{C}'>0$, noting Lemma \ref{below}
and the fact that $X$ was assumed
to be conditionally $L$-mixing of order $(2,4)$. We conclude that
$$
E^{1/2}| \theta_{t}-\overline{z}_{t}|^2\leq C^{\sharp}\lambda \sqrt{T}
|\ln(T)|^{3/2}\leq C^{\star}\sqrt{\lambda}|\ln(\lambda)|^{3/2},
$$
with some $C^{\sharp},C^{\star}>0$, for all $t\in\mathbb{N}$.

Now we turn to
estimating $|\overline{z}_t-\overline{\theta}_t|$. By Lemmata \ref{ghibli}
and \ref{easy},
\begin{eqnarray*}
||\overline{z}_t-\overline{\theta}_t||_2 &\leq &\\
\sum_{k=1}^n ||z(t,kT,\theta_{kT}) - z(t, (k-1)T, \theta_{(k-1)T})||_2 &=& \\
\sum_{k=1}^n ||z(t,kT,\theta_{kT}) - z(t, kT, z(kT,(k-1)T, \theta_{(k-1)T}))||_2
&\leq &\\
\sum_{k = 1}^{n} e^{-\lambda\rho(t-kT)/2}
\left[||\overline{\theta}_{kT -1} - \bar{z}_{kT-1}||_2 +
\lambda ||H(\overline{\theta}_{kT-1},X_{kT}) - h(\overline{z}_{kT-1})||_2\right]
&\leq &\\
\frac{C^{\dagger}}{1-e^{-\rho/2}}\left[1+C^{\flat}\right]\sqrt{\lambda}
|\ln(\lambda)|^{3/2},
\end{eqnarray*}
for some $C^{\dagger}>0$.
This completes the proof of the theorem since
$$
|\theta_t-\overline{\theta}_t|\leq |{\theta}_t-\overline{z}_t|+
|\overline{z}_t-\overline{\theta}_t|.
$$
\end{proof}








\section{Examples}\label{examples}

\section{The bright future}

I think that we can significantly generalize the above results using another approach, based on
\cite{eberle}. We will use the metric
$$
w(\mu,\nu):=\inf_{\zeta\in\mathcal{C}(\mu,\nu)}\int_{\mathbb{R}^d}\int_{\mathbb{R}^d} [1\wedge |x-y|](1+V(x)+V(y))\zeta(dx,dy).
$$

Consider the diffusion process $L_t$, $t\in\mathbb{R}_+$ defined by
$$
dL_t=-h(L_t)\, dt+ dB_t,\ L_0:=\theta_0.
$$
We will also need, for each $\lambda>0$,
$$
L^{\lambda}_t:=L_{\lambda t},\ t\in\mathbb{R}_+.
$$

Let us introduce the It\^{o} process
$$
dY^{\lambda}_t=-H(Y^{\lambda}_t,X_{\lfloor t\rfloor})\ dt+dB_t,\ Y^{\lambda}_t:=\theta_0.
$$

A new approach would consist of the following steps:

\begin{enumerate}

\item We only assume dissipativity and Lipschitz-continuity of $h(\theta),H(\theta,x)$ (the latter uniformly in $x$).
This guarantees that $L_t$ is contractive in the metric $w$, by \cite{eberle} and by e.g. \cite{unadjusted}.

\item The arguments presented above can equally well be performed in continuous time and
provide $w(\mathrm{Law}(L^{\lambda}_t),\mathrm{Law}(Y^{\lambda}_t))\leq C\sqrt{\lambda}|\ln(\lambda)|^{3/2}$,
for each $t$. (OK, strictly speaking the above argument contains $L^2$-estimates of the type $E|X-Y|^2$ but these
could be ameliorated to get estimates of the form $E|X-Y|(1+V(X)+V(Y))$ since $V$ is a Lyapunov function.)

\item Now the arguments of \cite{unadjusted} (which go back to Dalalyan) could be used to establish
the bound
$$
||\mathrm{Law}(Y^{\lambda}_t)-\mathrm{Law}(\theta^{\lambda}_t)||_{V}\leq C\sqrt{\lambda}
$$
on the weighted total variation norm
There is also $X_t$ intervening here but I think that the same arguments (using Kullback-Leibler
divergence) should work.

\item As $w(\cdot,\cdot)\leq C||\cdot-\cdot||_{V}$, this leads to a rate of convergence much better than
that of Raginsky, not in $W_2$, but in $w$. AND WITHOUT CONVEXITY OF ANY SORT! Also, $h$ does not need to
be the derivative of something for 2. and 3. to work.

\end{enumerate}

With Huy we could do 2., I think, while the other part of the team could do 3.
The machine learning community will tremble :).

\begin{thebibliography}{00}

\bibitem{4} N. H. Chau, Ch. Kumar, M. R\'asonyi and S. Sabanis.
\newblock On fixed gain recursive estimators with discontinuity in the parameters.
\newblock arXiv:1609.05166v3, 2017.

\bibitem{durmus-moulines} A. Durmus and \'E. Moulines.
\newblock High-dimensional Bayesian inference via the Unadjusted
Langevin Algorithm.
\newblock arXiv:1605.01559, 2018.

\bibitem{unadjusted} A. Durmus and \'E. Moulines.
\newblock Nonasymptotic convergence analysis for the unadjusted Langevin algorithm.
\newblock \emph{Ann. Appl. Probab.}, 27:1551--1587, 2017.

\bibitem{eberle} A. Eberle, A. Guillin and R. Zimmer.
\newblock Quantitative Harris-type theorems for diffusions and
McKean-Vlasov processes. \newblock\emph{Preprint.}, 2017.
\newblock arXiv:1606.0612v2

\bibitem{laci1} L. Gerencs\'er.
\newblock On a class of mixing processes.
\newblock \emph{Stochastics},  26:165--191, 1989.

\bibitem{neveu} J.Neveu.
\newblock\emph{Discrete-parameter martingales.}
\newblock North-Holland, 1975.

\bibitem{raginsky}
M. Raginsky, A. Rakhlin, and M. Telgarsky.
\newblock Non-convex learning via Stochastic Gradient Langevin Dynamics: a nonasymptotic analysis.
\newblock \emph{Proceedings of Machine Learning Research}, 65:1674--1703, 2017. \newblock arXiv:1702.03849

\bibitem{villani} C. Villani.
\newblock \emph{Optimal transport. Old an new.}
\newblock Springer, 2009.

\bibitem{xu} P. Xu, J. Chen, D. Zhou and Q. Gu.
\newblock Global convergence of Langevin dynamics based
algorithms for nonconvex optimization.
\newblock arXiv:1707.06618, 2018.

\end{thebibliography}


\end{document}


